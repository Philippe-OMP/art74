%                                                                 aa.dem
% AA vers. 8.2, LaTeX class for Astronomy & Astrophysics
% demonstration file
%                                                       (c) EDP Sciences
%-----------------------------------------------------------------------
%
%\documentclass[referee]{aa} % for a referee version
%\documentclass[onecolumn]{aa} % for a paper on 1 column  
%\documentclass[longauth]{aa} % for the long lists of affiliations 
%\documentclass[rnote]{aa} % for the research notes
%\documentclass[letter]{aa} % for the letters 
%\documentclass[bibyear]{aa} % if the references are not structured 
% according to the author-year natbib style

% V3

%
\documentclass{/Users/art2/TeX/aanda/aa} 
%\documentclass[referee]{/Users/art2/TeX/aanda/aa} 
%\documentclass{aa}
%\documentclass{article}
%
%\usepackage{graphicx}
\usepackage{graphicx}
%\usepackage{psfig}
%%%%%%%%%%%%%%%%%%%%%%%%%%%%%%%%%%%%%%%%
\usepackage{txfonts}
%\usepackage{longtable}
%%%%%%%%%%%%%%%%%%%%%%%%%%%%%%%%%%%%%%%%
\bibpunct{(}{)}{;}{a}{}{,} % to follow the A&A style
%\usepackage[options]{hyperref}
% To add links in your PDF file, use the package "hyperref"
% with options according to your LaTeX or PDFLaTeX drivers.

\def\kms {km\,s$^{-1}$}

\setlength{\LTcapwidth}{16cm}
\begin{document} 


   \title{The height of convective plumes in the Red SuperGiant $\mu$ Cep\thanks{Based on observations obtained at the T\'elescope Bernard Lyot
(TBL) at Observatoire du Pic du Midi, CNRS/INSU and Universit\'e de
Toulouse, France.}}
   %\subtitle{Velocities}

   \author{{ A.~L{\'o}pez Ariste}\inst{1},{ M.~Wavasseur}\inst{1}, { Ph. Mathias}\inst{1},  { A. L\`ebre}\inst{2}, { B. Tessore}\inst{3}, { S.~Georgiev}\inst{2,4}}
   \date{Received ...; accepted ...}

   
   \institute{IRAP, Universit\'e de Toulouse, CNRS, CNES, UPS.  14, Av. E. Belin. 31400 Toulouse, France \and
   LUPM, Universit\'e de Montpellier, CNRS, Place Eug\`ene Bataillon, 34095 Montpellier, France \and
   Universit\'e Grenoble Alpes, CNRS, IPAG, 38000 Grenoble, France \and
   Institute of Astronomy and NAO, Bulgarian Academy of Science, 1784 Sofia, Bulgaria 
%\institute{IRAP - CNRS UMR 5277. 14, Av. E. Belin. 31400 Toulouse. France 
  %  \and Universit\'e de Toulouse, UPS-OMP, Institut de Recherche en Astrophysique et Plan\'etologie, Toulouse, France}
}

% \abstract{}{}{}{}{} 
% 5 {} token are mandatory
 
  \abstract
  % context heading (optional)
  {}
  % {} leave it empty if necessary  
   {We seek understanding convection in red supergiants and  the mechanisms that trigger the mass loss from cool evolved stars.}
  % aims heading (mandatory)
   {Linear spectropolarimetry of the atomic lines of the spectrum of $\mu$ Cep reveals information well outside the  wavelength range expected from previous models. This is interpreted as structures in expansion visible in the front hemisphere and sometimes also in the back hemisphere. We model the plasma distribution together with its associated velocities through an inversion algorithm to fit the observed linear polarization.
 }
  % methods heading (mandatory)
   {We find that supposing the existence of plasma beyond the limb rising high enough to be visible above it can explain the observed linear polarization signatures as well as its evolution in time. From this we are able to infer geometric heights of the convective plumes and establish that this hot plasma raises to at least 1.1 $R_*$. }
     % results heading (mandatory)
   {$\mu$ Cep appears to be in an active phase in which plasma rises often above 1.1 $R_*$ . 
   We generalize this  result to all red supergiants in a similarly evolved stage which at certain epochs may easily send plasma at greater heights, as $\mu$ Cep appears to be doing at present. Plasma rising so high can easily escape the stellar gravity. }
  % conclusions heading (optional), leave it empty if necessary 
  
   \keywords{}

\titlerunning{Raising plumes in $\mu$ Cep}
\authorrunning{A. L\'opez Ariste et al.}

   \maketitle
%

\section{Introduction}

Despite their large impact on stellar and galactic evolution, the properties of outflows from Red Super Giants (RSG) are not well characterized. In particular, the role of convection is still poorly understood, partly because their structures are difficult to observe directly through interferometry. In this work, we propose a view of the convection structure of the RSG $\mu$ Cep through images reconstructed from spectropolarimetric data. Behind the interest on convection lays also the study of the increased mass loss of these evolved stars to which these convective processes are thought to contribute.


%Betelgeuse is quite often the favorite target for the study of red supergiants (RSG). Its large angular diameter   \citep[42.11 mas in the K band,][] {montarges_dimming_2021}, and its relative proximity (about 200\,pc) justify this choice. In this work, however, we focus on another prototypical RSG, $\mu$ Cep. The study of other red supergiants may help understanding the physics behind the convective phenomena in these large evolved stars by illustrating how the process changes with different mass, age or metallicity. 

%Behind the interest on convection lays also the study of the increased mass loss of these evolved stars to which these convective processes are thought to contribute.

These have been the motivations for several campaigns of observation of the red supergiant $\mu$ Cep at the Telescope Bernard Lyot (TBL) with both spectropolarimeters Narval and Neo-Narval. These observations have been made in parallel to those of Betelgeuse presented by  \cite{Auriere_2016}, \cite{Mathias:2018aa} and \cite{LA18}, though at smaller frequencies and not so systematically. But as for Betelgeuse, the observations of $\mu$ Cep have measured in priority the linear polarization in the atomic lines of its spectrum. 

The observed net  linear polarization has been interpreted as the depolarization, during line formation, of the spectrum continuum which is itself polarized by Rayleigh scattering \citep{Auriere_2016}. Using this hypothesis of the physical origin of the observed linear polarization, 2-dimensional images of the photosphere of Betelgeuse were inferred, images that could be favorably compared to co-temporal images of this star made by interferometric techniques \citep{LA18}. These successful comparisons give plausibility to the many approximations involved in this new imaging technique using spectro-polarimery and have spurred a further step in the technique: recently \cite{LA22} published the first  3-dimensional images of Betelgeuse.  Beyond the satisfying images, this technique offered some measurements of interest: the spatial and temporal scales of the convective patterns have been measured, and the characteristic velocities of the raising plasma were determined. These velocities are much higher than the adiabatic estimates, reaching easily values of 40\kms \citep{LA18,stothers_giant_2010}. More interestingly, \cite{LA22} observed that in  several observed cases of raising hot plasma, this velocity was constant with height, suggesting the presence of a force counter-acting gravity in the photospheric layers. These large velocities, sometimes reaching 60\kms, are comparable to the escape velocity at tantalizing low heights ($1.5 R_*$). If at any time it reaches the escape velocity, this plasma will escape the star's gravity and, cooling down, it may be the origin of the clumpy dust clouds seen around Betelgeuse \citep{montarges_dimming_2021}. This fast raising plasma may be the source of mass loss in these stars. 

But in order to reach this interesting result, a critical information is missing. The technique presented by \cite{LA22} to build 3-dimensional images of the photosphere of Betelgeuse is unable to determine the geometric height of the successive layers imaged. The technique only provides the ordering of the layers, from the deep atmosphere up to higher layers, but not the geometric distance among them. The possibility to measure or, at least, estimate this geometric height is what we present in this work. 


In Section 2 we present the set of observations of $\mu$ Cep collected  with Narval and Neo-Narval at the TBL from 2015 through 2022.
 In Section 3, we describe the spectral features in the linear polarization of $\mu$ Cep that cannot be explained with the model used to image Betelgeuse, and propose a modification of the model. We propose that, from time to time, convective plumes are powerful enough to raise sufficiently high so that they can be seen beyond the geometric horizon of the star.  This cannot be a permanent feature, but it may happen from time to time, an aspect that is critical to make this proposition plausible. We discuss also how this modification affects previous results of Betelgeuse, if we assume that a unique model serves all red-supergiants.
In Section 4 we build an inversion code based upon this modified model, where the usual description of the brightness variation across the disk is supplemented with the presence of up to 5 clouds of plasma visible beyond the horizon of the star. The measured linear polarization degrees and angles allow us to determine how far beyond the horizon this plasma is and therefore, how high it must be to be visible from Earth. It is in this way that we can determine a minimum height for these structures. In Section 5, we build a time evolution of one of those plumes that we were lucky to follow  in 2021 from rise to fall. In the last Section we put these measurements in context, in particular with  respect to the measured plasma velocities. We confirm that it is highly possible that the most powerful of these convective plumes are high enough to escape the star's gravity at the observed velocities. 
 
 
\section{Spectropolarimetric data from  Narval and Neo-Narval}
 

$\mu$ Cep is an M2-type RSG with stellar parameters ($T_{eff} = 3750 K$ and $\log g = -0.36$ ) very similar to those of Betelgeuse, while however  its mass (25$M_\sun$)  and radius (1420 $R_\sun$) may be larger \citep{levesque_physical_2005}. \cite{tessore_measuring_2017}  first detected strong linear polarization features (both in Stokes $Q$ and  $U$) associated to atomic lines.

We have observed  $\mu$ Cep in linear polarimetry since July 2017 with the Telescope Bernard Lyot at Pic du Midi (France,TBL). Until August 2019, the Narval spectropolarimeter was used. After an upgrade, starting in September 2019,  Neo-Narval re-observed $\mu$ Cep in May 2020 and regular observations have been conducted since then. This long series allowed us to follow the full life of one of these convective plumes (see Sect. 5).


Narval and Neo-Narval have been described before in the literature. Extensively for the case of Narval \citep{Donati2006}, and with a sober description of the changes  of Neo-Narval by \cite{LA22}.  As this last reference stresses, we continue here in noticing the continuity of data quality from the instrument through its upgrade, and handle Narval and Neo-Narval data as a unique dataset with no further reference to the instrument used.

We performed quite short exposures (about 3minutes per polarimetric sequence) in order to ensure a peak signal-to-noise ratio (S/N) of about 2000 in Stokes I per velocity bin.
A list of  the observations of $\mu$ Cep  is presented in Table \ref{tab1}, corresponding to all those studied in this work. An LSD procedure \citep{Donati2006} is applied to the reduced spectra. Atomic lines from an appropriate list \citep{Auriere_2016} are added up after rescaling of the wavelength binning. The result is a single spectral profile for both Stokes I and the observed Stokes parameter.
The whole set Stokes Q, U  and I profiles thus obtained is  shown on Fig.\ref{velos} in the form of an image with time in the vertical dimension.%In such a long time series, with at best one profile per week, observations would be barely visible if limited to the actual date of observation. To make the image illustrative we have decided to assign the same profiles to the fifteen days following an observation, unless a new one is available. This increases the vertical size of the profiles and makes then visible. Through this artefact it is also  apparent that while changing, profiles are coherent over several weeks. 

\section{Red-shifted linear polarization features.}
The model used to interpret the linear polarization observed in the atomic lines of Betelgeuse and $\mu$ Cep and, in general, of all red supergiants assumes a non-rotating convective star. This model and the implicit approximations involved have been described in detail by \cite{LA18} and \cite{LA22}. From the point of view of the physical origin of the linear polarization, our model assumes that what we observe is the depolarization by atomic lines of the continuum due to Rayleigh scattering. A key diagnostic to trust this interpretation of the polarization is that all lines must show similar polarization independent of their quantum structure. In particular the Na $D_1$ and $D_2$ lines must show similar signals to one another. This has been seen to be the case for Betelgeuse \citep{Auriere_2016}, CE Tau, and now for $\mu$ Cep,  the target of the present study. Once the physical  origin of this polarization confirmed,  the model focuses on the distinct spatial origin of the spectral features seen in the linear polarization profiles.

The observed linear polarization profiles characteristically show several distinctive lobes inside every atomic line. In the absence of rotation, the wavelength position of each of those lobes is assumed to be due to convection. The brightest plasma is assumed to be rising, and the cooler, darker plasma sinks. At first approximation, most light comes from the brighter regions and is therefore Doppler shifted by the projection onto the line of sight of the convective velocity at which it rises. Thus, bright hot plasma at disk center will emit light in the blue wing, while bright hot plasma at the limb will emit light in the red wing, at a wavelength which will coincide with the velocity of the center of mass of the star with respect to the Sun. Dark, sinking plasma would be redshifted with respect to this red wing, but its low intensity translates into a  tiny signal  to be added in the further red wing of the observed profile. In this model the spectral profile of an atomic line is framed by two velocities. One of these velocities is the heliocentric velocity $V_*$, that limits the red wing of the polarization profile. 
Plasma at the stellar limb will emit light at or near to this red wavelength. The second of these velocities is the maximum velocity of the plasma in the convective flow, $V_p$, that limits the blue wing of the profile. Plasma raising at this maximum velocity  at disk center will emit light at the bluest wavelengths.  In the absence of rotation, all other velocity fields, such as micro and macro-turbulent velocities or thermal broadening,  are assumed to be isotropic and would just broaden the signals. Such broadening, added to instrumental effects, is seen as a minimum width for all observed polarization signals, a width which is much smaller than the span of velocities attributed to convection.

The two velocities, $V_*$ and $V_p$, that limit the observed profiles are parameters of the model and should be determined a priori. As discussed by \cite{LA22} this a priori determination is done by inspection of the whole set of available observations. In Fig. \ref{velos} the two velocities are represented as vertical lines on top of the pile up of the observed profiles. Our choice for these two velocities: $V_*=+35$ \kms for the velocity of the center of mass of the star, and $V_p=-70$ \kms for the maximum velocity of the convecting plasma can be judged with respect to the wavelength span of the polarization signals. Notice that while $V_*$ is measured in the heliocentric reference system, we are giving the value of $V_p$ in the star's own reference system. In the heliocentric reference system used in Fig. \ref{velos} we find $V_p$ at $+35-70=-35 $\kms. Since $V_p$ has a meaning in terms of the physics of convection of the star, it is useful to keep its value in the reference system of the star, even at the risk of some confusion when looking into Fig.\ref{velos}.

 The choice of these values is not free from criticism. Judging from Fig.\ref{velos} alone, it appears as if the red limit $V_*$ has been placed in the middle of the polarization signal rather than at its red edge.  Since these two velocity limits cannot be directly measured, we can only advance the arguments that justify our choice for these two parameters.   These arguments are qualitatively similar to the ones used by \cite{LA18} and that \cite{LA22}
 justified to be acceptable within 10 \kms. Part of this justification lays on the fact that accepting the model and the values of these velocities results in images of Betelgeuse or CE Tau, another observed RSG, which are comparable with contemporaneous images inferred by interferometers \citep{LA18}.  In the present case, however, we lack any such interferometric images for $\mu$ Cep, and  contrary to the previously studied Betelgeuse and CE Tau, there is a considerable amount of signal to the red of $V_*$. It is worth to examine the arguments that justify this choice.

 It is obvious that $V_p$, the maximum velocity of the plasma represented by the blue line  in Fig.\ref{velos} (which, remember, is in the heliocentric reference system, while the value of $V_p$ is given in the star's reference system) must encompass the most blue-shifted signals observed over the years. Accordingly, added to $V_*$, this velocity must be somewhere beyond -20\kms in Fig.\ref{velos}.  We have chosen -35\kms to include the extended wings of the signals observed.  In our model we have no explanation whatsoever for any signal to the blue of $V_p$. We have to make sure therefore that there is no signal beyond this limit, and this fixes minimum values of $V_p$.

In our model, $V_p$ is interpreted as the velocity of the raising plasma during convection in the reference frame of the star. This interpretation sets further constraints on its maximum value. Our choice has been, for $\mu$ Cep to set this maximum velocity at $V_p=-70$ \kms. This is already a large velocity for raising plasma. It is roughly seven times the speed of sound in the atmosphere of $\mu$ Cep. Numerical simulations \citep{freytag_spots_2002, chiavassa_radiative_2011} produce supersonic flows  for convective patterns, confirmed by \cite{LA18}. But  a value of  $V_p=-70$ \kms is 
50 to 100\% larger than any announced figure so far, either observationally or numerically.

But, since $V_p$ is fixed on its blue side by the extent of the polarization signal, accepting these large plasma velocities is the only manner of placing $V_*$  as far to the red as possible.
Since $V_*+V_p$ is set at -35 \kms,  the velocity of the center of mass must therefore be  $V_*=+35$\kms.  
In spite of the large value of $V_*$ this choice  still leaves lots of red-shifted signal beyond the limit of the model. Once again, including those signals in our convection model by shifting $V_*$ to higher values would imply accepting convection velocities larger than 70\kms and this seems unphysical. And once more, diminishing the maximum velocity $V_p$ seems also unphysical, since blue-shifted signatures would be left unexplained beyond the maximum velocity of our model. This sequence of arguments justifies, up to 10\kms our choices, and leaves large amounts of signal beyond the red limit of $V_*$
 
 Another argument justifying the choice of  $V_*=35$\kms is apparent from  the Stokes $I$ variation.
From the LSD profile, one can compute a mean heliocentric radial velocity from the profile gaussian fit, with an amplitude of $<v>=22$\kms of the center of the intensity line in the heliocentric reference frame, with an amplitude of variation of about 4\,km\,s$^{-1}$.
For Betelgeuse, the corresponding quantities are respectively $<v>=21$\kms and 4\kms .
The radial velocity of Betelgeuse was estimated to be about $V_*=40$\kms.
Supposing the same parameters for both these RSGs i.e., similar order of granules number, temperature contrast, then the
$V_* - <v>$ values should also be similar, therefore a value of the order of 40\kms  or, within
the 10\kms uncertainty, the adopted $V_*=35$\kms value.

To conclude this discussion about the choice of the value of these two velocities, we must add that several choices of velocities have been tested inside the range allowed by those 10 \kms, without significant differences in the results. 

\begin{figure}
\includegraphics[width=0.5\textwidth]{FigN.png}
\caption{Pile-up of the Stokes Q (left), U (center) and I (right) profiles over the whole time series. For illustrative purposes, every observation has been made to span 15 days on the vertical direction. The blue and red vertical lines mark the maximum plasma velocity $V_p$ and the radial velocity of the center of mass of the star $V_*$ respectively (see main text for definitions). Velocities are measured in the heliocentric reference system. }
%\includegraphics[width=0.5\textwidth]{mucep_polarization_peaks_limits_v2.png}
%\caption{Velocity position (upper plot) and span (lower plot) of the linear polarization spectral features of $\mu$ Cep over the whole data set of available observations. On the top plot, dots record the position of the signal peaks, while in the bottom plot the vertical lines show the span of signal above noise level. The two horizontal lines in both plots mark the velocity of the center of mass of the star $V_*$ (red line, at +35 \kms) and the maximum velocity of the convecting plasma $V_p$ (blue line, at -35\kms). Velocities are measured in the heliocentric reference system.}
\label{velos}
\end{figure}



A strict interpretation of these  velocity limits  implies that no polarization signal can be seen  in our modeled profiles at wavelengths redder than $V_*$, the limit given by the velocity of the center-of-mass of the star for each atomic spectral line. Polarization signals beyond this limit would come, in this model, from plasma moving away from the observer, towards the center of the star. Plasma sinking towards the core of the star is assumed to be cool, dark plasma. This rigorous interpretation must be softened somehow, as dark cold plasma emits still some light and plasma may start sinking while still being bright enough to contribute to the net spectral line profile. But these are always small contributions. We are not expecting any large signals on the red side of the velocity limit. Inspection of the profiles of Betelgeuse published by \cite{Auriere_2016,Mathias:2018aa,LA18} and \cite{LA22} confirms that this is the case, and that the hypothesized model can confidently describe all the available observations.  This is however not the case for $\mu$ Cep. 

In  Fig. \ref{prof1} we plot observed spectra of $\mu$ Cep collected since September 6th, 2015. The observations are plotted in dotted lines. We will come back to the continuous lines in different colors later on. At this point, our attention focuses on the strong Q signal peaking at about +50 \kms. This is actually the strongest polarization of the observed profiles on that date and it is found to the red of the limiting velocity  $V_* =+35$ \kms, indicated by the vertical dashed line.
\begin{figure*}
\includegraphics[width=\textwidth]{Fig1.png}
\caption{Observed linear polarization of $\mu$ Cep on September 6th, 2015. The observed Stokes Q is plotted at left, and Stokes U at right as dots. Continuous lines represent the best fit from the assumed model (green line) with separated contributions from the front disk brightness distribution (red) and the two plumes beyond the limb (black), visible at those wavelengths when they do not constitute the whole contribution to the final fit in green. The upper (orange) profile shows the normalized intensity profile. The vertical dashed lines give the two limiting velocities, $V_p$ and $V_*$.}
\label{prof1}
\end{figure*}

Such strong signal in the red wing of the profiles of atomic lines is unlike anything observed  to this day in Betelgeuse. The many more observations of  linear polarization in the spectra of Betelgeuse are better illustrated by the profile of Stokes U, shown in  the right plot of Fig.\ref{prof1}. A strong peak is seen in the blue wing and attributed to bright plasma near the center of the stellar disk, another (negative) peak is seen near $V_*$, and attributed to bright  raising plasma coming from regions near the stellar limb, and a small   (positive) signal is seen beyond $V_*$ corresponding hypothetically  to sinking dark plasma. The observed Stokes U profile is in this manner qualitatively explained and, after inversion, the inferred image confirms this basic description of the visible structures. Such model would also explain the small lobe seen in the blue wing of the Stokes Q profile and the larger negative peak near the red boundary (red line of Fig. \ref{prof1}). The respective amplitudes and signs of these peaks in Q and U will constrain the position and brightness of the different bright structures over the disk. But this model has no explanation whatsoever for the strong signal on the red side of the red boundary of the Q profile. Such strong signal cannot be attributed to dark sinking plasma, for there would be no explanation for its large amplitude. The amplitude can be due to either the amount of photons, or the polarization degree of those photons. To interpret such a large amplitude would require either a brighter region or a more polarized region. It appears as contradictory to say that the sinking plasma is brighter than the raising plasma, so we are only left with the possibility that this is sinking plasma with an enormous polarization degree. Implicit in our model is that polarization degree is directly related to the height of the plasma. So one possible explanation that our model would have for this strong polarization peak would be that this is a huge cloud of cold plasma sinking from large heights, much larger than any other structure in the atmosphere, since height must compensate the loss of signal due to the lower emissivity of this cool dark plasma.

The presence of that unexpected strong peak is forcing the model towards extreme scenarios.  

Another possibility is that our determination of $V_*$, the velocity of the center-of-mass of the star, is wrong. It suffices to shift this limit a further 35 \kms to the red, up to $V_*=+70$\kms, and the peak will entirely fit inside the limits of the model. But this scenario  brings up unpleasant conclusions too. Shifting this limit to the red  without touching the blue limit would mean that the maximum velocity of the convective flows in $\mu$ Cep would increase to a staggering 100 \kms. This is an uncomfortably large number for the convective flows, about 10 times the speed of sound. The problems with a modified red boundary do not end here. We expect always some signal coming from near the limb, since statistically there is a large probability of finding a bright structure somewhere along the long circumference. This is the case with the actual red boundary limit plotted in Fig.\ref{prof1}: both Stokes Q and U profile show signal near the limb. However if we accept to shift this boundary, Stokes Q would still have the strong peak that would be attributed to a near-limb structure, but no comparable signal is visible any near that limit for Stokes U. In order to produce such signal imbalance between Q and U, one would need to imagine a stellar disk with a continuous dark band along and inside the limb except at one position where a bright structure would give the observed Stokes Q signal. While not impossible, this appears as a strange disposition of structures on $\mu$ Cep, something never seen on Betelgeuse. In addition, this 70 \kms value is clearly outside the I profile, meaning that this latter would have no link with the heliocentric star velocity, which seems, also, difficult to accept.


\begin{figure*}
\includegraphics[width=\textwidth]{Fig2.png}
\caption{Observed linear polarization of $\mu$ Cep on October 21st, 2016. Same color codes for Stokes Q and U as in Fig\ref{prof1}. The velocity boundaries given by the dashed lines are common to all the observations of $\mu$ Cep.}
\label{prof2}
\end{figure*}

One year later, in October 2016, the observations, shown in Fig.\ref{prof2}, have drastically changed. Between the velocity boundaries, the polarization signals keep providing an image of changing bright, convective, structures. But there is always signal at the qualitatively expected places, even if that signal has changed in amplitude, ratio and position. This is interpreted as bright structures that have moved over the disk, some have appeared anew, others disappeared. But there is always signal coming from around disk center and visible around the blue wing, and signal coming from the limbs and visible around the red wing but on the correct side of the boundary $V_*$. The big change is that at this date, and contrary to the observations in 2015, there are no conspicuous signals on the wrong, red, side of $V_*$. There are always small amplitude signals, both in $Q$ and $U$. Because of their small amplitude they can be comfortably assigned to dark sinking plasma, or perhaps a small error in the determination of $V_*$, an error of at most 10 \kms consistent with the rough arguments used to its determination. But there is no large peak visible throwing doubts on the validity of the model.

These two observations of $\mu$ Cep show an expected signal between the velocity boundaries that, while changing, is always there. It can be explained as it was explained in Betelgeuse: by a spatially inhomogeneous distribution of bright patterns that have been interpreted as convection. But they also show a new signal  that appears and disappears in time and that, if we accept the velocity boundaries, corresponds to bright plasma moving away from the observer.


The conclusions drawn from  these qualitative arguments are definitively confirmed by the inversion codes developed by \cite{LA18} and \cite{LA22}: the model used to fit the observed polarized spectra of Betelgeuse 
and to infer the published images is unable to produce a solution for the spectrum of $\mu$ Cep on September 2015, though it provides a solution for the observations of October 2016. Unwilling to drop a model that has been successful with Betelgeuse, we propose an addition to this model that can explain the intermittent appearance of strong signals on the red side of the red velocity boundary, as those illustrated in Fig. \ref{prof1}. We propose that the bright convective structures inferred for Betelgeuse and $\mu$ Cep and present over the whole star, also in the back hemisphere,  may raise plasma high enough for it to become visible above the stellar limb. We refer as plumes to this high rising hot plasma.
When these plumes are on the front hemisphere, they produce the signals between the two velocity boundaries and the basic model is able to explain them. Similar convective bright structures must occur also on the back hemisphere, but they are usually hidden by the stellar limb. From time to time, one of these bright structures in the back hemisphere may push plasma high enough for it to become visible to us above the limb. This plasma raises with a radial direction, but since it is in the back hemisphere of the star we see it red-shifted, moving away from us, beyond the red velocity boundary. It is bright plasma nevertheless, so we expect it to have similar polarization amplitudes to plasma in the front hemisphere in symmetric geometries. Usually plasma is not supposed to raise high enough, so we often expect to see nothing beyond $V_*$. This has been the case for all available observations of Betelgeuse and also for $\mu$ Cep on October 2016. But from time to time this may happen producing the signal illustrated in Fig.\ref{prof1}.  When this is observed, it cannot happen all over the stellar limb, but only at particular polar angles, thus explaining the single peak that is visible only in Stokes $Q$.    

Becoming visible over the limb depends geometrically on the distance to that limb of the bright structure.  The further a structure is from the limb, the higher it has to raise to become visible.  This suggests that we can determine the height of one of those structures as the minimum height at which, geometrically, it becomes visible above the limb.  This measurement of a minimum height for the raising plasma is going to be our main result. 


%Plasma near the limb, either in the front or back hemispheres, easily raises at sufficient heights. But since our red velocity boundary is not perfectly defined, there is a region around the limb for which we cannot tell with present data if the structure is on the front or the back hemisphere: it all depends on the precise but unknown value of the red velocity boundary.  
\section{Inversions with a modified model}

In accordance to the suggested modification of the model proposed in the last section upon inspection of those polarization signals beyond the red velocity boundary, we have built an inversion code to fit the observed spectra of $\mu$ Cep. The core of this inversion code is identical to the one described by \cite{LA18}. Mathematically it is a Marquardt-Levemberg algorithm that fits the observed Stokes Q and U profiles with synthetic profiles computed from a distribution of brightness over the surface of  the star. On the front hemisphere, this distribution of brightness is described by a linear combination of spherical harmonics up to sixth order. The blue velocity boundary is the maximum velocity of the raising plasma. The brightest point over the disk at any particular realization of the model is supposed to move radially at that maximum velocity. All other points over the disk have a brightness described by the spherical harmonics, and a velocity which is mathematically relied to its brightness so that the resulting brightness contrast and velocities roughly match the solar case \citep[see the Appendix in][]{LA22}. The polarization emitted by a point over the hemisphere is proportional to that brightness, but also to the squared sine of the scattering angle, as expected for Rayleigh scattering. The ratio of polarizations between Q and U is given by the tangent of half the polar angle position of the point. Its wavelength is determined by its velocity projected onto the line of sight and thus it depends on the distance to the center of the disk. Mathematically, the model uses as parameters  the coefficients of  a brightness distribution written in terms of spherical harmonics as
\begin{equation}
B(\mu,\chi)=\left \| \sum_{\substack{\ell=0,\ell_{max} \\ m=-\ell,+\ell}} a_{\ell}^m y_{\ell}^m(\mu,\chi) \right \|
\label{Beq}
\end{equation}
with $\ell_{max}=6$, and $\mu$ and $\chi$ the angle to disk center and the polar angle with respect to celestial north respectively. This brightness distribution results in the emission of net polarization described in terms of Stokes parameters as
\begin{eqnarray}
Q_{disk}(v)=\sum_{\mu,\chi,v_z} B(\mu,\chi) \sin^2 \mu \cos 2\chi e^{-(v-v_z)^2/\sigma^2}\\
U_{disk}(v)=\sum_{\mu,\chi,v_z} B(\mu,\chi) \sin^2 \mu \sin 2\chi e^{-(v-v_z)^2/\sigma^2}
\label{QU}
\end{eqnarray}
where $v_z=V(\mu, \chi) \cos \mu$, with $V(\mu, \chi)$ the plasma velocity at that point, proportional to the brightness and  limited by the maximum speed of the plasma $V_p$. Each emission is broadened with a gaussian profile of fixed width $\sigma=6$ \kms  (i.e. 10 \kms FWHM) representing both instrumental and thermal broadenings.
This is a quick description of the basic model, for which much more details are given and scrutinized in the \cite{LA18} and \cite{LA22}.
On top of  this basic model, we assume the presence of one or several sources of polarization beyond the limb. When adding new parameters to the model to describe those new sources of polarization one should be careful not to overload the inversion algorithm with more new unknowns than available new information. Thus it is out of question to try to provide a description of the continuous distribution of brightness in the back hemisphere, since only a very limited amount of that plasma will be contributing to the observed spectra. It is tempting to try to propose a description of the brightness in a ring above the limb.  Unfortunately, we have not found a proper mathematical description for such a ring. One of the difficulties is that since we assume that raising high enough to be visible above the limb is not common, we are expecting contributions from at most a small range of polar angles, the rest contributing zero. Any orthogonal family of functions trying to describe this paucity of sources requires a too large amount of parameters. We have finally opted for a simplistic description in terms of a small amount of discrete sources. Each one of the discrete sources over the limb is described by its polar angle $\chi$, its angular distance to the limb $\theta$ and a brightness value $Z$ (see cartoon in Fig.\ref{cartoon}). Its polarization is given, as in the case of any other emitting point in the front hemisphere, by the scaled product of its brightness and the squared sine of the scattering angle, this scattering angle being geometrically related to the distance to the limb.
\begin{eqnarray}
Q_{\mathit{off}}(v)=\sum_{i=0,N} Z_i \sin^2 \theta_i \cos 2\chi_i e^{-(v-v_z)^2/\sigma^2}\\
U_{\mathit{off}}(v)= \sum_{i=0,N} Z_i \sin^2 \theta_i \sin 2\chi_i e^{-(v-v_z)^2/\sigma^2}
\label{QU_off}
\end{eqnarray}

\begin{figure}
\includegraphics[width=0.5\textwidth]{Cartoon.png}
\caption{A cartoon defining the parameters of a discrete source (yellow sphere) in the back hemisphere (in grey) beyond the plane of the sky (bluish plane). Celestial north is up, in the plane of the sky. The image of the front hemisphere corresponds to the inferred brightness distribution of $\mu$ Cep on September, 2015.}
\label{cartoon}
\end{figure}

The radial velocity of the raising plasma is identically given as a function of brightness.  Its red-shifted wavelength is analogously given by the projection of this velocity onto the line-of-sight, a projection which is once more geometrically dependent on the distance to the limb. Each one of these discrete over-the-limb sources produces a polarization peak in Stokes Q and U which is broadened by a Gaussian profile with full-width at half-maximum (FWHM) of 10 \kms. This FWHM is supposed to encompass both instrumental resolution and various stellar broadening mechanisms, thermal, micro-turbulent and so on.  Finally, both sources of polarization, $Q_{disk}, U_{disk}$, and $Q_{\mathit{off}}, U_{\mathit{off}}$ are added.

The last parameter to be determined is the number $N$ of discrete sources to be allowed. We have found unpractical to leave this number unbound. We have preferred to fix it. From $N=0$ up to $N=5$ discrete sources were attempted. Obviously, having zero sources allows us to recover the basic model, unable to reproduce the anomalous signals on the red wing. On the other end, we have found that beyond four sources we are not learning anything new from the inversion results, but the algorithm becomes unstable, and presents convergence issues. This can be safely understood as the number of new parameters being too high compared to the available information.   Between one and four sources is therefore the right number of sources that we can safely infer. Interestingly, we also found that for  any individual  observation of our long dataset, the inferred value of the polar angle of all the sources was similar, even if the intensity and height of each one of them was different. This means that the solution found by the inversion algorithm proposes that at a given polar angle there are several sources of polarization at different heights and with different intensities. That is, the inferred sources clump together on the same region above the limb. This can be interpreted as one single, but extended, source  over the limb of $\mu$ Cep at the time of the observation. This is a result comforting the intuition that such events of high-raising plasma are not common.  From this conclusion one may expect that one single source in our model would be sufficient to describe the observed polarization profiles in the red wing. But we found that this is not the case and that we needed a minimum of two sources to reproduce the basic spectral features observed. This may be indication that even if there is a unique object beyond the limb, it has sufficient structure that our description in terms of a gaussian profile per source is inadequate.  Using two or more sources becomes a simple manner of better describing the extent and structure of the emitting region. Because of this result, we present in this work inversions with just 2 sources. This has the advantage of capturing the important physical parameter for our work, the main distance of the bright structure to the limb,  while easing the constraints on the inversion algorithm. The observed structure is  often spectrally broader than twice the FWHM of 10 \kms of every discrete source. The fit is therefore somehow approximative. By increasing the number of sources we will improve this fit, but will not bring any further information.   

All this is illustrated in the already presented Figs \ref{prof1} and \ref{prof2}. Both figures show on top of the observed profiles the solution found by the inversion code as a green continuous line. This solution is made of three different contributions. The basic model describing the front hemisphere as a linear combination of spherical harmonics is plotted in red. It is fully coincident with, and hidden behind,  the full solution between the two dashed lines that limit the contribution of the front hemisphere. It can only be seen as a tail of small signal on the red side of the red velocity boundary. This small signal, as said above, is the contribution from the dark sinking plasma, insufficient to explain the observed polarization peak on September 2015, but almost sufficient to explain all the redshifted signal on October 2016. The two other contributions combined are shown as a black continuous curve, and correspond to two discrete sources above the limb. Again, this black line is only visible when it does not fully coincide with the final solution, plotted in green color.
As explained, limiting the number of sources to just two results in an approximative fit of the redshifted signal. The full solution profile clearly shows two peaks on the red wing, coinciding with the maxima of the two sources, a feature absent in the observations. There is also a clear tail further towards the red in the observations that cannot be captured with just two sources. Adding more sources would correct these missed fits, but the parameters of the added sources will not change much. On September 2015, the two sources over the limb bring signal comparable to anything else over the front disk. On October 2016, the two sources appear as small contributions that may drop to zero if just the red velocity boundary was shifted a few \kms towards the red. Thus the modified model is able to capture both those cases with important sources over the limb as well as those cases with negligible contributions.

We have inverted with this model using two sources above the limb the whole available dataset of linearly polarized spectra of $\mu$ Cep presented in Sect. 2. Imaging from linear spectropolarimetry is subject to a certain number of ambiguities: Several images, with different distributions of brightness are compatible with the same observed polarized spectra,
that is, they are possible alternative solutions of the inversion problem. These several images are not completely unrelated. The most common ambiguity shows two images identical up to a 180 degrees rotation. Comparison with images of Betelgeuse made with interferometric techniques allows us to determine which of these two rotated images is the one that better corresponds to the reality. But we do not have interferometric images for all dates, and none for $\mu$ Cep. Because of this, in the case of Betelgeuse, the best solution for a date with available interferometric images is propagated as the initial solution to the next date, thus encouraging the inversion code to stay in the group of solutions sharing choices among the ambiguities possible that better compared to interferometric images at one particular date. Similarly, for $\mu$ Cep, we have inverted the first available date without constraints. But for next dates, the solution of the previous available date was used as initial condition. This ensures a certain time coherence in the series of images. 

The inversion code provides  values for the polar angle and distance to the limb of the two sources along time. The distance to the limb $\theta$ is directly converted into minimum height $h$ above the stellar surface for this source  in the back hemisphere to be visible above the limb. 
\begin{equation}
h=\frac{1-\cos \theta}{\cos \theta} R_{*}
\end{equation}
Presented in this manner, the results of our inversions are shown in Fig. \ref{height}


\begin{figure*}
\includegraphics[width=\textwidth]{Fig3_trans.png}
\caption{Plots of  the height of the two sources visible above the limb and of their polar angle for the observations of  $\mu$ Cep of the last six years. For heights below 0.05, the source is considered to be absent and the corresponding value of the polar angle is made transparent.}
\label{height}
\end{figure*}

Over the last six years, $\mu$ Cep appears to have produced three events in the back hemisphere with plasma being lifted at considerable heights. The first of these events was undergoing when our observations started in September 2015. It had completely disappeared when the star was re-observed in late spring 2016. The next event started one year later, between January and April 2017, and by January 2018 plasma had reached heights of  at least 1.1$R_*$ and perhaps higher. This plasma appeared at polar angles of 100 degrees and after the winter blind window, the plasma was still at the same position and at even greater heights of 1.15$R_*$. Over the spring of 2019 the emitting plasma was seen to be at lower and lower heights, until it disappeared by the summer of that year.  The beginning of the observations with Neo-Narval  at the beginning of  2020 showed $\mu$Cep to be still quiet, with no particular signals on the red wing. But this situation changed  by the end of the year with the rapid rise of a new clump of plasma at polar angle 0 degrees, therefore unrelated to the previous one, which in less than one month reached heights of at least 1.175$R_*$. The maximum height reached by this event appears to be quite ephemeral. As fast as it rises, it disappears. But as it disappears, we are left with a low  lying clump that persists over 2022. Optimistically, we may interpret this as a large event of rising plasma inside of which there is a small clump at high speed reaching even higher heights in a short time before disappearing, perhaps due to a quick cooling, while the rest of the raising plasma is still visible.  In all these events, the value of the polar angle of the two sources is quite similar, as can be seen in the right plot of Fig. \ref{height}. As said above, we interpret this result as proof that there is a unique source above the limb but more extended and complex than what our model with two gaussians can reproduce. 

Our eight-years-long observations of linear polarization of Betelgeuse have not produced any single event sufficiently large to require a modification of the inversion model. In 5 years of observations, $\mu$ Cep  has produced 3 such events. Is this due to the slightly different stellar parameters of these two stars? The fundamental parameters of $\mu$ Cep recently determined by \cite{montarges_noema_2019} show a star similar to Betelgeuse within  error bars. Rather than calling for fundamental differences between the two stars, we speculate that $\mu$ Cep may be at present in a \textit{Decin stage} \citep{decin_probing_2006}, as suggested by  \cite{montarges_noema_2019}, with common episodes of mass loss, while Betelgeuse may rather be in a quiet stage with rare and separated such events. This is obviously just a speculation. At this point we lack any clear scenario of why and when a red supergiant may enter into a \textit{Decin stage}, if such episodes happen to exist at all. Further observations in time will be needed  to see if $\mu$ Cep stops producing or not these events\footnote{It must be said that    \cite{decin_probing_2006} estimate the duration of such episodes in the tens of years.}. 

In the fall of 2019, Betelgeuse suffered a large dimming that has been attributed to the formation of a dense dust cloud almost right along our line of sight \citep{montarges_dimming_2021}. \cite{LA22} suggested that these mass loss events are triggered from fast raising plasma in the photosphere reaching the escape velocity at a certain height. Their suggestion stemmed from the measurement of plasma velocities constant with height and sufficiently large to be comparable to escape velocities at the estimated heights of these structures. It is tempting to see this event in Betelgeuse as one example of the more common events in $\mu$ Cep of plasma raising sufficiently high to be visible above the limb. But in the case of Betelgeuse, such event happened in the front hemisphere, rather than in the back hemisphere as in $\mu$ Cep. If we accept that Betelgeuse is at present in a quiet stage of mass loss, unlike $\mu$ Cep, events where plasma is ejected from the star appear to still happen.
Just  by chance,  in Betelgeuse, lately, they have not been happening in the regions near the limb, but rather in the front disk, the ultimate example being the one that produced the large dust cloud involved in the great dimming of 2019. In $\mu$ Cep on the other hand, 3 such events have taken place in regions around the limb, making it visible to our spectropolarimetric measurements.

 


\section{Follow-up of a convective plume above the limb}

\begin{figure*}
\includegraphics[width=0.5\textwidth]{Fig4_1.png}
\includegraphics[width=0.5\textwidth]{Fig4_2.png}
\includegraphics[width=0.5\textwidth]{Fig4_3.png}
\includegraphics[width=0.5\textwidth]{Fig4_4.png}
\includegraphics[width=0.5\textwidth]{Fig4_5.png}
\includegraphics[width=0.5\textwidth]{Fig4_6.png}
\caption{Time series of spectropolarimetric observations of $\mu$ Cep corresponding to all dates from September 15, 2020 and thru May 1, 2021 showing the rise and fall of a convective plume. Meaning of curve colors and styles is the same as in Fig.\ref{prof1}.}
\label{fall}
\end{figure*}

Figure \ref{fall} shows a time series of spectropolarimetric observations of $\mu$ Cep starting on September 15, 2020 and ending in May 1, 2021. The first 5 observations during the fall and winter of 2020 show the rapid raise of a convective plume above the celestial north limb of the star. It can easily be identified in the inferred heights shown on Fig. \ref{height}. Such behavior can also be seen directly in the profiles  as a red peak with negative amplitude in Stokes Q that, date after date, shifts to redder and redder wavelengths. This means that its projection over the line of sight is greater and greater. Our interpretation of this is that plasma beyond the limb is raising. First, the parts nearer to the limb became visible above the limb and, as time goes on, plasma farther and farther from the limb become visible as they reach the height at which this is geometrically possible. The plume, centered well beyond the limb is raising over a period of 3 months. We loose track of the star from January through April and in the first observation in May the structure has almost completely disappeared: the polarization beyond the red velocity boundary is small and centered very near the limit, as if only the regions closer to the limb were still emitting light. The plume has disappeared.  We have chosen this event to illustrate how the rise of the plume can be guessed from direct visual inspection of the profiles, before the inversion code confirms the interpretation. The rise of the plume is quite fast. And similarly fast is its disappearance, since at the opening of the observing window in the next spring, there is barely any signal of its presence. 

It may be tempting to say that the plume fell back into the star, but we have no signature of this. We must recall that, any plasma falling back into the back hemisphere would produce a blue-shifted signal that would melt into the signals of raising plasma of the front hemisphere. We have no manner to disentangle both origins of polarization. 

The event of 2018 is better followed during its disappearance. What we observe is that the signal is still well visible in the red wing, meaning that the emitted plasma is still raising in the back hemisphere. But its height is lower and lower. We interpret this as follows: The upper parts of the plume of plasma, while still raising, stop emitting light in the atomic lines measured. This may be because as it cools down, its brightness diminishes, or because atomic lines are no more excited. Translating our technique to molecular lines, if feasible, would shed light on this. In one and the other case, the top of the plume cools down first and stops emitting light. We only measure light from lower parts, still hot enough and still raising. This process keeps going on, until only the lower parts of the plume are emitting measurable signals. Therefore, at the end of these episodes, we  do not see the plume falling, but just disappearing from our sensing window of atomic lines in what we interpret as a cooling down that starts from the top of the plume.

\section{Discussion on the height of the observed structures} 


Looking back into Fig.\ref{height} we see that the structure followed in Fig.\ref{fall} reached a minimum height of $1.175R_*$ during those 3 months. Perhaps higher since by geometry we can only give the lower bound of this height. Taking as radius of the star $1000 - 1200 R_{\odot}$ this raise required an average velocity of 15 to 20 \kms, unchecked for 3 months. This velocity fits comfortably with the velocity limit $V_p=70$\kms  in $\mu$ Cep determined for the basic model of the front hemisphere. These are just minimum velocities, since we can only determine minimum heights.

The possibility of detecting this kind of plumes over the limb, offered by $\mu$ Cep, is an exceptional chance. The observation of three such events in over five years may suggest that there is some exceptional convective activity in this star, at least when compared with Betelgeuse. Nevertheless we stick to the assumption that both stars represent different cases of the same physics and that it is just the relative short spans of observations available which explains the observed differences, and not any fundamental difference between the phenomena in these two RSG. Building on this assumption,  the measurement of a geometric height made on these structures is deemed typical of  convective plasma features in RSG, and we generalise it to all other structures imaged on RSG with spectropolarimetry. We consider the  value of $1.1R_*$  as the typical height of the plasma in the atmospheres of RSG hot enough to emit atomic spectral lines. Making the link with \cite{LA22}, we consider that this measured geometric height must correspond typically to the height of their uppermost layer, recovered from spectropolarimetry of the deepest atomic lines in the spectrum. Their Fig.9 must extend therefore up to $1.1R_*$. It is only by considering that this upper layer is visible during several months at that height that in that work it is assumed that the observed structures may well have reached $1.3R_*$. 

\section{Conclusion}

Spectropolarimetric observations of the red supergiant $\mu$ Cep show spectral features in linear polarization that were not observed in the better studied Betelgeuse. Such spectral features are much more redshifted than any other signal and are not permanent features: at some particular dates the observed spectra are qualitatively identical to those of Betelgeuse. We argue that the origin of those unexpected spectral features is convective plumes in the back hemisphere of the star raising high enough to be visible above the stellar limb. 

This hypothesis allows us to conserve the inversion algorithms and model that have successfully been applied to Betelgeuse and that have produced images comparable to those from interferometry. But such basic model must be extended to allow for temporary sources of polarized light above the stellar limb. We have produced an inversion algorithm using this extended model and successfully fit the observed profiles, included the unexpected new features. This model assumes the presence of a small number of discrete sources over the limb. Although about four such sources are required to correctly fit the profiles, we realize that at any given date all those sources appear to be combined to describe a unique but extended source on the star. This observation has allowed us to reduce the number of sources over the limb to just two. While the fit with just two sources is not as good as it would with four sources, we still capture the main parameters of the sources and stabilize the convergence of the code which can be automatically launched to handle the whole dataset available.

The inversion results produce the polar angle position and the height of those sources. This gives us access, for the first time, to a geometric height for the convective structures detected through spectropolarimetry. 

Three events of plasma rising over the limb have been observed during the six years of observation of $\mu$ Cep with Narval and Neo-Narval at the TBL at Pic du Midi. Two of those events have been tracked during their rapid rising phase and into their disappearance. The characteristic heights reach  $1.1R_*$ and even $1.175R_*$ in the last observed event. We consider this a typical value of the heights of the convective structures observed in the photosphere of RSG, and \cite{LA22} use this value to put a geometric height to the 3-dimensional images of Betelgeuse. Thanks to this measurement they demonstrated that the measured velocities in the plasma are very near the escape velocity of Betelgeuse and that this raising plasma must be a probable contributor to the mass loss of these stars.

\begin{acknowledgements}
This work was supported by the "Programme National de Physique Stellaire" (PNPS) of CNRS/INSU co-funded by CEA and CNES.
S.G. acknowledges support under the Erasmus+ EU program for doctoral mobility. S.G. acknowledges partial support by the Bulgarian NSF project DN 18/2.
\end{acknowledgements}

\bibliographystyle{/Users/art2/TeX/aanda/bibtex/aa}
%\bibliographystyle{aa}

\bibliography{art72}

 \begin{appendix}
 

 \section{Log of Observations}


 \begin{table} [h]
 \caption{Log of Narval  and Neo-Narval  observations of $\mu$ Cep and polarimetric measurements since July 2015.}
 \begin{tabular}{lcl}%{p{3cm}p{3cm}p{1cm}p{3cm}p{3cm}p{3cm}}%{lccccc}

\hline 
\hline
Date &  Julian date & Stokes \\
&&Sequence \\
 \hline
%\endfirsthead
%\caption{continued}\\
%\hline
%\hline
%Date &  Julian date & Stokes \\
%&&Sequence \\
%\hline
%\endhead
%\multicolumn{3}{r}{\textit{Continued on next page}} \\
%\endfoot
%\endlastfoot
July 10, 2015 & 7214.578 & 8U+8Q \\
September 05, 2015 & 7271.504 & 2U+2Q \\
November 10, 2015 & 7337.462 & 2Q+2U \\
May 16, 2016 & 7525.609 & 2Q+2U \\
June 08, 2016 & 7548.628 & 2U \\
June 16, 2016 & 7556.604 & 2Q \\
June 21, 2016 & 7561.586 & 2U \\
June 27, 2016 & 7567.569 & 2Q \\
July 05, 2016 & 7575.566 & 2Q+2U \\
July 15, 2016 & 7585.517 & 2Q+2U \\
August 02, 2016 & 7603.448 & 2U+2Q \\
September 01, 2016 & 7633.418 & 2Q+2U \\
September 28, 2016 & 7660.436 & 2Q+2U \\
October 07, 2016 & 7669.366 & 4Q+2U \\
November 27, 2016 & 7720.273 & 2Q+2U \\
December 18, 2016 & 7741.265 & 4Q+4U \\
January 07, 2017 & 7761.276 & 2Q+2U \\
April 08, 2017 & 7852.658 & 2Q+2U \\
April 15, 2017 & 7859.664 & 2Q+2U \\
April 22, 2017 & 7866.669 & 2Q+2Q \\
May 07, 2017 & 7881.614 & 2Q+2U \\
May 20, 2017 & 7894.562 & 2Q+2U \\

June 01, 2017 & 7906.57 & 2Q+2U \\

June 11, 2017 & 7916.545 & 2Q+2U \\

June 16, 2017 & 7921.578 & 2Q+2U \\

July 02, 2017 & 7937.627 & 2Q+2U \\

July 11, 2017 & 7946.616 & 2Q+2U \\

July 31, 2017 & 7966.561 & 2Q+2U \\

August 08, 2017 & 7974.536 & 2Q+2U \\

August 13, 2017 & 7979.45 & 2Q+2U \\

August 21, 2017 & 7987.524 & 2Q+2U \\

September 02, 2017 & 7999.445 & 2Q+2U \\

September 05, 2017 & 8002.412 & 2Q+2U \\

September 13, 2017 & 8010.449 & 2Q+2U \\

September 20, 2017 & 8017.504 & 2Q+2U \\

September 27, 2017 & 8024.37 & 2Q+2U \\

October 02, 2017 & 8029.469 & 2Q+2U \\

October 07, 2017 & 8034.342 & 2Q+2U \\

October 12, 2017 & 8039.359 & 2Q+2U \\

October 30, 2017 & 8057.341 & 2Q+2U \\

November 07, 2017 & 8065.285 & 2Q+2U \\

November 14, 2017 & 8072.337 & 2Q+2U \\

November 19, 2017 & 8077.255 & 2Q+2U \\

November 26, 2017 & 8084.269 & 2Q+2U \\

December 04, 2017 & 8092.275 & 2Q+2U \\

May 17, 2018 & 8256.624 & 2Q+2U \\

June 14, 2018 & 8284.607 & 2Q+2U \\

June 30, 2018 & 8300.503 & 2Q+2U \\

July 22, 2018 & 8322.619 & 2Q+2U \\
\hline
\end{tabular}
\end{table}
 \begin{table} [t]
\begin{tabular}{lcl}%{p{3cm}p{3cm}p{1cm}p{3cm}p{3cm}p{3cm}}%{lccccc}

\hline 
\hline
Date &  Julian date & Stokes \\
&&Sequence \\
 \hline
 


August 13, 2018 & 8344.677 & 2Q+2U \\

September 26, 2018 & 8388.478 & 2Q+2U \\

October 25, 2018 & 8417.442 & 2Q+2U \\

November 14, 2018 & 8437.322 & 2Q+2U \\

December 10, 2018 & 8463.313 & 2Q+2U \\

January 04, 2019 & 8488.239 & 2Q+2U \\

January 16, 2019 & 8500.281 & 2Q+2U \\

March 21, 2019 & 8564.685 & 2Q+2U \\

May 05, 2019 & 8609.594 & 2Q+2U \\

June 01, 2019 & 8636.633 & 2Q+2U \\

June 18, 2019 & 8653.622 & 2Q+2U \\


July 18, 2019 & 8683.555 & 2Q+2U \\

August 02, 2019 & 8698.589 & 2Q+2U \\

August 15, 2019 & 8711.498 & 2Q+2U \\

August 30, 2019 & 8726.552 & 2Q+2U \\

January 06, 2020 & 8855.288 & 2Q+2U \\

May 17, 2020 & 8987.572 & 1Q+2U \\

June 22, 2020 & 9023.621 & 2Q+2U \\

July 05, 2020 & 9036.583 & 2Q+2U \\

July 24, 2020 & 9055.576 & 2Q+2U \\

August 22, 2020 & 9084.518 & 2Q+2U \\

September 15, 2020 & 9108.483 & 2Q+2U \\

October 16, 2020 & 9139.313 & 2U+2Q \\

November 21, 2020 & 9175.329 & 2Q+2U \\

December 18, 2020 & 9202.256 & 2Q+2U \\

January 13, 2021 & 9228.271 & 2Q+2U \\

May 01, 2021 & 9337.647 & 2Q+2U \\

May 26, 2021 & 9361.603 & 2Q+2U \\

June 14, 2021 & 9380.536 & 2Q+2U \\

July 10, 2021 & 9406.619 & 2Q+2U \\

August 07, 2021 & 9434.455 & 2Q+2U \\

August 19, 2021 & 9446.59 & 2Q+2U \\

September 04, 2021 & 9462.43 & 2Q+2U \\

October 06, 2021 & 9494.441 & 2Q+2U \\

October 13, 2021 & 9501.405 & 2Q+2U \\

November 09, 2021 & 9528.363 & 2Q+2U \\

December 13, 2021 & 9562.279 & 2Q+2U \\

December 22, 2021 & 9571.252 & 2Q+2U \\

January 11, 2022 & 9591.264 & 2Q+2U \\
\hline
\end{tabular}
\label{tab1}

\textbf{Notes:} Columns give the date, the heliocentric Julian date (+2\,450\,000),  and the observed Stokes sequence, that is, how many observations of which Stokes parameter were made at that date. An observation consists upon 4 exposures with changing polarimetric modulation that, after reduction, produce polarization spectra of either Stokes $Q$ or $U$. Beyond a 2-year proprietary embargo, all data is publicly available at PolarBase (http://polarbase.irap.omp.eu/).

\end{table}

 
 
 
 
 

 
 \end{appendix}

\end{document} 
