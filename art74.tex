%                                                                 aa.dem
% AA vers. 8.2, LaTeX class for Astronomy & Astrophysics
% demonstration file
%                                                       (c) EDP Sciences
%-----------------------------------------------------------------------
%
%\documentclass[referee]{aa} % for a referee version
%\documentclass[onecolumn]{aa} % for a paper on 1 column  
%\documentclass[longauth]{aa} % for the long lists of affiliations 
%\documentclass[rnote]{aa} % for the research notes
%\documentclass[letter]{aa} % for the letters 
%\documentclass[bibyear]{aa} % if the references are not structured 
% according to the author-year natbib style

% V3

%
\documentclass{/Users/art2/TeX/aanda/aa} 
%\documentclass[referee]{/Users/art2/TeX/aanda/aa} 
%\documentclass{aa}
%\documentclass{article}
%
%\usepackage{graphicx}
\usepackage{graphicx}
%\usepackage{psfig}
%%%%%%%%%%%%%%%%%%%%%%%%%%%%%%%%%%%%%%%%
\usepackage{txfonts}
%\usepackage{longtable}
%%%%%%%%%%%%%%%%%%%%%%%%%%%%%%%%%%%%%%%%
\bibpunct{(}{)}{;}{a}{}{,} % to follow the A&A style
%\usepackage[options]{hyperref}
% To add links in your PDF file, use the package "hyperref"
% with options according to your LaTeX or PDFLaTeX drivers.

\def\kms {km\,s$^{-1}$}

\setlength{\LTcapwidth}{16cm}
\begin{document} 


   \title{Making sense of Betelgeuse and other RSG spectra\thanks{Based on observations obtained at the T\'elescope Bernard Lyot
(TBL) at Observatoire du Pic du Midi, CNRS/INSU and Universit\'e de
Toulouse, France.}}
   %\subtitle{Velocities}

   \author{{ A.~L{\'o}pez Ariste}\inst{1}, { Q.~Pilates}\inst{1},{ A.~Lavail}\inst{1},{ Ph. Mathias}\inst{1},  }
   \date{Received ...; accepted ...}

   
   \institute{IRAP, Universit\'e de Toulouse, CNRS, CNES, UPS.  14, Av. E. Belin. 31400 Toulouse, France 
%\institute{IRAP - CNRS UMR 5277. 14, Av. E. Belin. 31400 Toulouse. France 
  %  \and Universit\'e de Toulouse, UPS-OMP, Institut de Recherche en Astrophysique et Plan\'etologie, Toulouse, France}
}

% \abstract{}{}{}{}{} 
% 5 {} token are mandatory
 
  \abstract
  % context heading (optional)
  {}
  % {} leave it empty if necessary  
   {}
  % aims heading (mandatory)
   {}
  % methods heading (mandatory)
   { }
     % results heading (mandatory)
   {}
  % conclusions heading (optional), leave it empty if necessary 
  
   \keywords{}

\titlerunning{Making sense of RSG spectra}
\authorrunning{A. L\'opez Ariste et al.}

   \maketitle
%

\section{Introduction}


\section{Spectropolarimetric data from  Narval and Neo-Narval}
 


\section{Red-shifted linear polarization features.}

% \begin{figure}
% \includegraphics[width=0.5\textwidth]{FigN.png}
% \caption{Pile-up of the Stokes Q (left), U (center) and I (right) profiles over the whole time series. For illustrative purposes, every observation has been made to span 15 days on the vertical direction. The blue and red vertical lines mark the maximum plasma velocity $V_p$ and the radial velocity of the center of mass of the star $V_*$ respectively (see main text for definitions). Velocities are measured in the heliocentric reference system. }
% %\includegraphics[width=0.5\textwidth]{mucep_polarization_peaks_limits_v2.png}
% %\caption{Velocity position (upper plot) and span (lower plot) of the linear polarization spectral features of $\mu$ Cep over the whole data set of available observations. On the top plot, dots record the position of the signal peaks, while in the bottom plot the vertical lines show the span of signal above noise level. The two horizontal lines in both plots mark the velocity of the center of mass of the star $V_*$ (red line, at +35 \kms) and the maximum velocity of the convecting plasma $V_p$ (blue line, at -35\kms). Velocities are measured in the heliocentric reference system.}
% \label{velos}
% \end{figure}





\section{Conclusion}


\begin{acknowledgements}
This work was supported by the "Programme National de Physique Stellaire" (PNPS) of CNRS/INSU co-funded by CEA and CNES.
\end{acknowledgements}

\bibliographystyle{/Users/art2/TeX/aanda/bibtex/aa}
%\bibliographystyle{aa}

\bibliography{art74}

 

\end{document} 
